\documentclass{myDocClass}
\graphicspath{{./images/}}
%************************************************************
%Lab nbr ?
\def\labN{2}
%Course
\def\course{Advanced Digital Systems}
% Report Title
\def\thetitle{Lab \#\labN}                   
% Date
\def\date{\Large\today}                           
% instructor full name
\def\instructor{Dr. A. benzekri}   
%author
\author{Zakaria Madaoui}

% DOCUMENT START
\begin{document}

% TITLE PAGE
\begin{center}
    % University Logo
    \includegraphics[scale = 0.7]{umbb.png}\\[1cm]
    % University Name
    \textsc{\LARGE{People's democratic republic of Algeria}}\\
    \textsc{\LARGE{University of M'hemed BOUGARA Boumerdes}}\\[0.5cm]
    \textsc{\color[RGB]{0, 51, 102}\LARGE{INSTITUTE OF ELECTRICAL AND ELECTRONICS ENGINEERING}}\\
    \rule{\textwidth}{2pt}\\
    \vspace{1cm}
    \textbf{\LARGE{\course}}\\[.5cm]
    \textsc{\LARGE{\thetitle}}\\[.5cm]
    \textsc{\date}\\[2cm]
    
    %authors area
    \Large{
    \begin{tabular}{L{5cm} R{5cm}}
        \textit{Authors}\\
        \hline
        % author names  and group
        Madaoui Zakaria & Group 2\\
        Lariane Sidali & Group 2
    \end{tabular}\\[1cm]
    % instructor area
    \Large\textbf{ Instructor:} \instructor
    }
\end{center}


\thispagestyle{empty}
\pagebreak

\changefontsize[15pt]{12pt}

% TABLE OF CONTENTS
\tableofcontents{}
\pagebreak

% REPORT START
% \section*{Background}
% \addcontentsline{toc}{section}{Background}

%intro
\section{Introduction}
\lipsum[1-1]

\section{Tools and Software}
\blindenumerate[4]

\section{Part I: bla bla bla}
\lipsum[1-1]

This is some VHDL code:

%direct code
\begin{lstlisting}[style=vhdl, title = "Title here"]
  -- import std_logic from the IEEE library
  library IEEE;
  use IEEE.std_logic_1164.all;
  entity ANDGATE is
    port ( 
      I1 : in std_logic;
      O  : out std_logic);
  end entity ANDGATE;
 
  architecture RTL of ANDGATE is
  begin
    O <= I1 ; -- this is a comment
  end architecture RTL;
  \end{lstlisting}

  % code from a file (much better, since you can have syntax highlighting this way)
  \lstinputlisting[language=VHDL, style=vhdl ,title = "Mealy FSM"]{codes/FSM_Mealy.vhd}

\section{Conclusion}
\lipsum[1-1]

\begin{figure}[H]
  \centering
  \begin{tikzpicture}[->,>=stealth',shorten >=1pt,auto,node distance=6cm, semithick , scale=.8,transform shape ]
  
\node[initial,state] (A)                    {wait command};
\node[state]         (B) [right of=A] {concat name};
\node[state, align=left]         (C) [right of=B] {concat\\parameter};
\node[state, align=left]         (E) [below of=C] {save \\parameter};
\node[state, align=left]         (D) [below of=A] {save command\\\& notify user};

\path (A) edge              node {charIn == `$>$'} (B)
        edge [loop above] node {charIn != `$>$'} (A)
    (B) edge [loop above] node {charIn != `,'} (B)
        edge              node {charIn == `,'} (C)
    (C) edge [loop above] node {charIn != `,'} (C)
        edge              node {charIn == `,'} (E)
        
    (E) edge [bend left]  node {charIn != `$<$'} (C)
        edge [bend left]  node {charIn == `$<$'} (D)
    (D) edge              node {} (A);
\end{tikzpicture}
\caption {Parsing command state machine}
\label{lst:helper_statemachine}
\end{figure}

%************************************************************
\end{document}



%************************ Common Code ******************


%% code for creating tables: https://www.overleaf.com/learn/latex/Tables

%%%% Include a picture
% \begin{figure}[H]
%   \centering
%   \includegraphics[width = 0.5\textwidth]{project_struc}
%   \caption{database entities relation}
% \end{figure}


%%%% Code for sub figures
% \begin{figure}[H]
%   \centering
%   \begin{subfigure}[b]{0.35\textwidth}
%       \centering
%       \fbox{\includegraphics[width=\textwidth]{ex_commands.jpg}}
%       \caption{Commands for this example}
%       \label{fig:ex1_commands}
%   \end{subfigure}
%   \hspace{1cm}
%   \begin{subfigure}[b]{0.35\textwidth}
%       \centering
%       \fbox{\includegraphics[width=\textwidth]{ex_gui.jpg}}
%       \caption{GUI for this example}
%       \label{fig:ex1_gui}
%   \end{subfigure}
%   \end{figure}